\chapter{Implementacja}

Założeniem projektu było stworzenie konfigurowalnej platformy umożliwiającej przeprowadzanie testów wydajnościowych dla różnych metod komunikacji międzyprocesowej. Głowna część (silnik testujący) stworzona została w oparciu o Javę 9 i jej wirtualną maszynę.


\section{TODO}

TODO


\section{tmp}

- testy na sucho (mock) w celu sprawdzenia narzutu samej platformy
- testy powatarzane wielokrotnie, aby uniknąć skrajnych wyników
- na początku niemierzone testy, aby rozgrzać JVM
- obrazek architektury klientów (może per metoda komunikacji?)
- może snippet kodu metody test z klasy AbstractTransferTester
- grupowanie danych po nanosekundach, req size, res size, test type
- wykorzystane biblioteki?
- sekcja per metoda komunikacji
- snippet generowania responsa używany w C/C++
