\chapter{Wprowadzenie}


https://en.wikipedia.org/wiki/Inter-process\_communication

wspoldzielony zasob:
File
Memory-mapped file

Signal; also Asynchronous System Trap	A system message sent from one process to another, not usually used to transfer data but instead used to remotely command the partnered process.




jeden system:
Message queue
Pipe
Shared memory


sieć:
Socket <- na nim kolejne protokoly (wlasny protokol na TCP, REST, websocket)
Message passing: distributed objects (CORBA), actor model (akka)
sync communication: REST


uruchomienie kodu natywnego (JNI)




TODO
Każde rozwiązanie ma swoje wady i zalety. Niektóre rozwiązania dziłają w architekturze klient - serwer, inne współdzielą zasoby lub wykonują potrzebny program. Czynników decydujących o wyższości nad innymi jest wiele. To przede wszystkim czas komunikacji dla różnych rozmiarów wiadomości, ale także wspierane systemy operacyjne, możliwość komunikacji sieciowej i wsparcie dla wielu języków programowania.
